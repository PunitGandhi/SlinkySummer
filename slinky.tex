\documentclass[aps,pre,10pt,superscriptaddress,showpacs,amsmath,amssymb,nofootinbib]{revtex4-1}

\usepackage{graphicx}
\usepackage{hyperref}
\usepackage{amsmath}
\usepackage{amssymb}
\usepackage[margin=1in]{geometry}
\usepackage{dcolumn}

\newcommand{\eq}[1]{eq.~\eqref{eq:#1}}
\newcommand{\eqs}[2]{eqs.~\eqref{eq:#1} and \eqref{eq:#2}}
\renewcommand{\sec}[1]{section~\ref{sec:#1}}
\newcommand{\secs}[2]{sections~\ref{sec:#1} and \ref{sec:#2}}
\newcommand{\subsec}[1]{section~\ref{subsec:#1}}
\newcommand{\subsubsec}[1]{section~\ref{subsubsec:#1}}
\newcommand{\app}[1]{appendix~\ref{app:#1}}
\newcommand{\fig}[1]{figure~\ref{fig:#1}}
\newcommand{\figs}[2]{figures~\ref{fig:#1} and \ref{fig:#2}}
\newcommand{\tab}[1]{table~\ref{tab:#1}}
\newcommand{\nn}{\nonumber}

\newcommand{\FIGstudents}{
\begin{figure}[t]\center
\includegraphics[width=\columnwidth]{FIGstudents.pdf}
\caption{\label{fig:students} Students doing stuff with slinkies}
\end{figure}
}


\newcommand{\FIGpulsedrop}{
\begin{figure}[t]\center
\includegraphics[width=\columnwidth]{FIGpulsedrop.pdf}
\caption{\label{fig:pulsedrop} A few frames from a slow motion image of a wave pulse sent down a slinky next to a falling slinky.  Possible a graph of position vs time of the top of the falling slinky and the wave pulse.}
\end{figure}
}


\begin{document}

\title{A tale of two slinkies: learning about model building in a student driven classroom }
\author{Punit  Gandhi}
\email{punit_gandhi@berkeley.edu}
\author{Jesse Livezey}
\author{Calvin Berggren}
\email{calvin1414@berkeley.edu}
\author{Ryan Olf}
\affiliation{Department of Physics, University of California at
Berkeley, Berkeley, California 94720, USA}
\date{\today}

\begin{abstract}
We describe a curriculum based around the dynamics of a vertically hanging slinky that is released from rest in an equlibrium position.
The motion, or lack thereof, of the bottom of the slinky serves as a counterintuitive phenomenon that provides context for learning about model building.
This set of conceptual activities and hands-on experiments was developed for a week-long summer program for incoming freshmen as a part of the Compass Project~\cite{albana2013}.  
This summer program is the first part of a three-course sequence aimed at helping students identify themselves as scientists.  
This curriculum, in particular, approaches this goal by giving the students the opportunity to actively collaborate in the process of developing models, exploring their limitations, and using this information to iterate on them.  
The falling slinky is an ideal tool for learning about model building because it allows the students to explore two distinct models that provide complimentary perspectives.
\end{abstract}

\maketitle

\section{Introduction}


The slinky drop experiment is supercool and suprises even experts who have never
seen the result before.  This gave us a way to "level the playing field" for
students with widely varying backgrounds    It has been studied in detail, cite
some of the references, and we use it as a way to contextualize the model
building process as we teach the students about models.

This curriculum was developed for an intensive one-week summer program for
incoming freshmen as a part of the Compasss Project ~\cite{albana2013}. some
details about compass?  The course was presented in a student-driven classroom
where the role of the role of the instructors were to facilitate/guide/provide
tools...  The important thing here is that the students developed the models
themselves, and took ownership over them.

The cirriculum is built around a central question that the students investigate.
 This central question actually evolves throughout the course of the week as
student gain understanding and develop a repetoir of scientific tools and
methods.  The question may start out as "Does the slinky defy gravity?" and end
up  "How does the bottom of the slinky know when to start falling?"

During the course of the program, we were able to approach the slinky in two different
and highly complementary ways. One approach, detailed in \sec{forces}, used forces and Newton's Laws to calculate
the motion of the slinky. In order to make this brute-force approach more tractable, we
modeled the slinky as a series of discrete masses connected by simple springs. This
model was conceptually very straightforward and allowed the students to follow an
intuitive reductionist approach where they could divide the slinky into individual
constituents and apply familiar rules to each part. Although the model turned out to
be formidable to solve, it was still highly beneficial as it was restricted to the
use of elementary physical concepts, which built up into a larger, more interesting
phenomenon. Students were able to see how the different parts interacted and how they
were interconnected.

Despite the conceptual simplicity of the force-based model, however, the explanation
required a large number of logical steps and a difficult calculation to arrive at
the final conclusion. For this reason, we also made use of a complementary
model based on information, detailed in \sec{information}. This approach attempted
to describe the event when the slinky was let go as a piece of information which
needs to travel to other parts of the system before they are able to respond. We
used waves the attempt to characterize the speed at which the information travels. 
As expected, information proved to be a much more sophisticated concept for the
students to grasp but provided a great deal of insight into the experiment.

The fact that the slinky drop lends itself so naturally to two very different models
makes it an excellent backdrop for teaching model-building. Having explored both
models, students are able to compare and contrast what kinds of insight are
offered by each one. Each model makes different parts of the underlying physics
more transparent. The great insight into the ``levitation'' provided by the
information-based model demonstrates the idea that the most useful models
capture only the effects that we are most interested in explaining. This curriculum
also shows how very different models can be successful describing the same
phenomenon. The models use different fundamental concepts and rules and approach
the problem in very different ways, yet obtain the same results. 

we did a exploritorium day with a bunch of activities to give students
experiences with all the concepts we would see in the week.  This gave shared
experiences for the students to draw from, and seeded some of the ideas that we
wanted them to have.

\section{Understanding gravity as it applies to the slinky}
A slinky was dropped next to an object at various heights relative to the
slinky.  The top of the slinky falls much faster than the object, but the bottom
falls much slower.  There is a point in the middle where the object and the
slinky hit the ground at the same time, this gives us a way give meaning to the
idea that gravity acts the same in some sense on the slinky as it does on
everything else. Namely the center of mass of the slinky falls as if it were a
point mass. An interesting extension is to ask the students to find the balance
point of a slinky if you could freeze it in the shape of the hanging slinky. 
This could be accomplished with rolled up paper on the inside of the slinky
along with lots of tape.



\section{Modeling the slinky using forces}
\label{sec:forces}

For this model, the slinky was represented as a series of discrete masses, each
connected to the next by a simple spring, as shown in \fig{discrete}. The strategy
was to find the forces (gravity and elastic) acting on each mass as a function of position and then
apply Newton's Laws to find the resulting motion. We first approached this model
by actually building such a series of discrete masses and taking measurements of
the motion (\subsec{forcesexperiment}). We then attempted to solve the equations
of motion numerically to find the motion (\subsec{forcesnumeric}).

\subsection{Experimentally testing the simplified model}
\label{subsec:forcesexperiment}
The simplified model can provide a path to analyze the effect of interest in the
full problem, but we would like to make sure that the simplified model also
captures the effect.  We can test this experimentally by building an experiment
that is exactly described by the simplified model and making sure we see what we
are looking for.  We had the students actually build model slinkies using
rubberbands and metal nuts.  It would be useful to redo the experiments here.
Actual measurements of the spring constant of the slinky, and that of the model 
should go here.  We also could define "Slinkiness" and have measurements of it
for various springs.


\subsection{Numerically testing the simplified model}
\label{subsec:forcesnumeric}
Having taken the step toward understanding the slinky using a model with
discrete masses separated by springs, we sought to make further progress by
solving this model numerically. The students spent some time constructing the
force equations for an $N=4$ system, which are
%%%
\begin{align} \label{eq:coupleddes}
m\ddot{x}_1 &= mg + k(x_2 - x_1)\,,
\nn\\
m\ddot{x}_2 &= mg + k(x_3 - x_2) - k(x_2 - x_1)
\,,\nn\\
m\ddot{x}_3 &= mg + k(x_4 - x_3) - k(x_3 - x_2)
\,,\nn\\
m\ddot{x}_4 &= mg                - k(x_4 - x_3)
\,,\end{align}
%%%
where $m$ is the mass of each discrete mass, $k$ is the spring constant of each
spring, $x_i$ refers to the absolute position of each mass, and the dots refer to
derivatives with respect to time.

Doing any serious exploration of these coupled differential equations
analytically was beyond the level of our audience; however, we briefly allowed
the students to consider how they would solve the system in order to demonstrate
the great difficulty of this strategy. We further emphasized the difficulty inherent in
solving coupled differential equations by showing a coupled pendulum demo where
one pendulum was suspended from another. The resulting semi-chaotic motion
showed that we would expect a complicated final answer.

Students were gradually led toward an alternative numerical approach to solving
the problem that divided the evolution of the system into small time
steps. This was motivated by considering examples like the frames in a
movie. Discretizing time expanded on our previous decision to approximate the
system by discretizing the mass in the slinky.

A major goal of this part of the curriculum was to show the benefits of
numerical simulation to solving problems. Not only do numerical approaches
practically provide a way to make progress solving equations that are difficult
to work with, but they also often provide a greater degree of transparency and
concreteness into how the system is evolving. We provided the class with the
ponderous analytical solutions to the 4-body system after the activity and asked
them how well they could see what was going on compared to the numerical
approach. We also asked how well they expected each approach to be able to
handle kinks or bends in the slinky to demonstrate the benefit numerical
approaches have in handling perturbations.

For the numerical approach, the students were asked to construct an algorithm
for how to proceed from one time step to the next. With the acceleration at each
step given by the force as in \eq{coupleddes}, the students settled on the
following simple algorithm (also known as the Euler method (Ref. \ref{..})),
%%%
\begin{align} \label{eq:algorithm}
v_i(t+\Delta t) &= v_i(t) + a_i(t)\Delta t
\,,\nn\\
x_i(t+\Delta t) &= x_i(t) + v_i(t)\Delta t
\,,\end{align}
with the time step given by $\Delta t$.

With the algorithm in hand, we had the students ``simulate'' the masses and
springs model as a group. Each mass was represented by a group of 
students, who completed the calculations for their mass to figure out where it would be
at the next time step. Students were divided into different roles, with some responsible 
for calculation, others for communicating their position to other groups who 
needed the information for their own calculations, and others for plotting the results on a graph 
at the front of the room. The communication of information between groups was intended 
to foreshadow the inherent time delay in the passing of information, explored later in
the week and described in \sec{information}.

Practical measures were definitely required for this activity to prevent the pace from dragging
too much. The groups were encouraged to streamline their work by assigning a specific
role to each person, working in parallel, creating an organized table of values and results,
etc. One potential problem is that, for an $N$-body simulation and for the algorithm described in \eq{algorithm},
it will take $2N$ time steps for the bottom mass to move. This is not an explanation for
the lack of motion in the real slinky but is rather an artifact of the discretization of time:
the last mass will always wait $2N$ time steps regardless of the size of the step. One way to
accelerate the activity would be to use a more accurate symplectic algorithm, such as the
symplectic Euler algorithm (Ref. \ref{...}). This algorithm would require only $N$ steps to move the bottom
mass, partially alleviating the tedium for the last group.

The participatory simulation described here provides good practice in building
a model by iteration. After beginning the activity, questions quickly arise, such
as what to do when a mass passes through another mass. Nothing in \eq{coupleddes} prevents
the masses from passing. The students may wish to allow this to happen as an approximation
or they may wish to modify their procedure, perhaps by merging any two masses who have overlaped
in the most recent time step. Follow up questions for discussion or investigation include
the dependence of the simulation on the size of the time step, the number of masses,
and other parameters.

It is also possible to perform this activity using a spreadsheet, which allows the steps to be
calculated much faster, albeit less collaboratively. We had the students implement the algorithm on a spreadsheet once they
felt comfortable with the details of the process. This allowed them to explore some of the follow
up questions more efficiently.

\section{Modeling the slinky using information}
\label{sec:information}

Realizing the limitations of the  mass and springs model helps motivate waves in
the slinky problem.  The numerical testing of the model provided an entry point
into discussion of information travel as student  physically relay the results
of calculations between the groups that represent the different masses.  It
becomes clear from the numerical simulation that the amount of time that the
bottom mass remains stationary depends on the size of the timestep.  The bottom
mass has to wait for the information from the top mass to trickle down to it. 
This motivates the study of waves as a mechanism for transferring information.

\subsection{understanding waves}
 We did a bunch of activities to introduce the basic concepts of waves, but I
 don't think any of them were particularly new.  It might be useful to talk about
dimensional analysis here in thinking about the wave speed.  We also did
a bunch of experiments to determine the dependence of the wave speed on
the parameters, it might be useful to redo these experiments to get actual data
for the slinkies.

\subsection{the wave pulse and the top of the slinky }
The final resolution to the levitation slinky comes when the students compare
the way the top of the slinky falls next to a wave pulse of a slinky that is
just suspended.  There are still many unanswered questions.

the students spend the last couple of days on final projects in groups where
they ask new questions or follow up on outsanding questions that we didn't have
time for during the week. include a few examples: torsional waves, center of 
mass, ...


\section{Discussion}
This section is probably not necessary.



\section{Conclusions}
This is an awesome cirriculum.  Compass is the best.  The summer program helps
our students identify as scientists.  maybe some short details about the other
two courses?


\acknowledgments The authors would like to acknowledge helpful discussions with
D. R. Dounas-Frazer and J. Corbo along with the support of the Compass Project
community.

\bibliography{slinky_bibliography}

\end{document}
