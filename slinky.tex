\documentclass[aps,pre,10pt,superscriptaddress,showpacs,amsmath,amssymb,nofootinbib]{revtex4-1}

\usepackage{graphicx}
\usepackage{hyperref}
\usepackage{amsmath}
\usepackage{amssymb}
\usepackage[margin=1in]{geometry}
\usepackage{dcolumn}



\newcommand{\FIGstudents}{
\begin{figure}[t]\center
\includegraphics[width=\columnwidth]{FIGstudents.pdf}
\caption{\label{fig:students} Students doing stuff with slinkies}
\end{figure}
}


\newcommand{\FIGpulsedrop}{
\begin{figure}[t]\center
\includegraphics[width=\columnwidth]{FIGpulsedrop.pdf}
\caption{\label{fig:pulsedrop} A few frames from a slow motion image of a wave pulse sent down a slinky next to a falling slinky.  Possible a graph of position vs time of the top of the falling slinky and the wave pulse.}
\end{figure}
}


\begin{document}

\title{A tale of two slinkies: learning about model building in a student driven classroom }
\author{Punit  Gandhi}
\email{punit_gandhi@berkeley.edu}
\author{Jesse Livezey}
\author{Calvin Berggren}
\email{calvin1414@berkeley.edu}
\author{Ryan Olf}
\affiliation{Department of Physics, University of California at
Berkeley, Berkeley, California 94720, USA}
\date{\today}

\begin{abstract}

\end{abstract}

\maketitle

\section{Introduction}
The slinky drop experiment is supercool and suprises even experts who have never
seen the result before.  This gave us a way to "level the playing field" for
students with widely varying backgrounds    It has been studied in detail, cite
some of the references, and we use it as a way to contextualize the model
building process as we teach the students about models.

This curriculum was developed for an intensive one-week summer program for
incoming freshmen as a part of the Compasss Project ~\cite{albana2013}. some
details about compass?  The course was presented in a student-driven classroom
where the role of the role of the instructors were to facilitate/guide/provide
tools...  The important thing here is that the students developed the models
themselves, and took ownership over them.

The cirriculum is built around a central question that the students investigate.
 This central question actually evolves throughout the course of the week as
student gain understanding and develop a repetoir of scientific tools and
methods.  The question may start out as "Does the slinky defy gravity?" and end
up  "How does the bottom of the slinky know when to start falling?"

The two models are useful because they allow for comparing and contrasting. The
masses and springs model actually tries to make a simplified model of the slinky
and attempts to capture the effect we are looking at.  We can only take this
model so far in gaining intuition for what we are after.    The wave model, is
less interested in modeling the general behavior of the slinky, but can give
great insight into the fact that the bottom seems to levitate.  This
demonstrates the idea that the most useful  models  capture only the effect that
we are after.

we did a exploritorium day with a bunch of activities to give students
experiences with all the concepts we would see in the week.  This gave shared
experiences for the students to draw from, and seeded some of the ideas that we
wanted them to have.

\section{Understanding gravity as it applies to the slinky}
A slinky was dropped next to an object at various heights relative to the
slinky.  The top of the slinky falls much faster than the object, but the bottom
falls much slower.  There is a point in the middle where the object and the
slinky hit the ground at the same time, this gives us a way give meaning to the
idea that gravity acts the same in some sense on the slinky as it does on
everything else. Namely the center of mass of the slinky falls as if it were a
point mass. An interesting extension is to ask the students to find the balance
point of a slinky if you could freeze it in the shape of the hanging slinky. 
This could be accomplished with rolled up paper on the inside of the slinky
along with lots of tape.



\section{Modeling a slinky with masses and springs}

This model is very well motivated by thinking of the slinky in terms of rungs,
and realizing how difficult it would be to model all the rungs.  Even with 2
masses and a spring, you can capture the effect that the bottom doesn't
initially move (to leading order).  However, no matter how many masses and
springs you use, the bottom will move before the top reaches it, so it doesn't
quite capture the effect we see.

\subsection{Experimentally testing the simplified model}
The simplified model can provide a path to analyze the effect of interest in the
full problem, but we would like to make sure that the simplified model also
captures the effect.  We can test this experimentally by building an experiment
that is exactly described by the simplified model and making sure we see what we
are looking for.  We had the students actually build model slinkies using
rubberbands and metal nuts.


\subsection{Numerically testing the simplified model}
It can be difficult to explore the model completely with experiments,  and it
may also be hard to take useful measurements.  We had the students "simulate"
the masses and springs model as a group. Each mass was represented by a group of
students, who completed the calculations to figure out where the mass would be
at the next time step.  We had to motivate the idea of discretizing time, which
we did with frames on a movie?  After this activity, we had the students
implement it with an excel spreadsheet.

This forced them to make certain decisions about the model and allowed them to
find several limitations.  what happens when the top passes the mass below it? 
what if the time step is too big?  should the time it takes for the bottom mass
to move depend on the size of the time step?


\section{Modeling the slinky drop experiment with wave propagation}

Realizing the limitations of the  mass and springs model helps motivate waves in
the slinky problem.  The numerical testing of the model provided an entry point
into discussion of information travel as student  physically relay the results
of calculations between the groups that represent the different masses.  It
becomes clear from the numerical simulation that the amount of time that the
bottom mass remains stationary depends on the size of the timestep.  The bottom
mass has to wait for the information from the top mass to trickle down to it. 
This motivates the study of waves as a mechanism for transferring information.

\subsection{understanding waves}
 We did a bunch of activities to introduce the basic concepts of waves, but I
 don't think any of them were particularly new or interesting where they?

\subsection{the wave pulse and the top of the slinky }
The final resolution to the levitation slinky comes when the students compare
the way the top of the slinky falls next to a wave pulse of a slinky that is
just suspended.  There are still many unanswered questions still.

the students spend the last couple of days on final projects in groups where
they ask new questions or follow up on outsanding questions that we didn't have
time for during the week.


\section{Discussion}




\section{Conclusions}



\acknowledgments The authors would like to acknowledge helpful discussions with
D. R. Dounas-Frazer and J. Corbo along with the support of the Compass Project
community.

\bibliography{slinky_bibliography}

\end{document}
